\documentclass{report}
\usepackage[utf8]{inputenc}
\usepackage[swedish]{babel}
\usepackage{url}
\usepackage{hyperref}

\title{Agenda 2030}
\date{\today}
\author{Johanna Sörbom}

\begin{document}
\maketitle
\newpage
\tableofcontents
\newpage
\section*{Sammanfattning}
\newpage
\section{Inledning}
Hållbar utveckling är ett viktigt samtalsämne i dagens samhälle och något som pushas hårt av bland annat FN. Hållbar utveckling definieras som utveckling som möter dagens behov utan att äventyra framtida generationers möjlighet att uppfylla sina.\cite{web2030agenda}
För att driva utvecklingen framåt på ett hållbart sätt som fungerar för alla världens länder har FNs medlemsländer tidigare kommit fram till millenniemålen som behandlar de viktiga punkterna för hållbarhet och utveckling.  Sedan millenniemålen skapades år 2000 har stora framsteg gjort i fattigdomsbekämpning, jämställdhet och utveckling. Men det finns mycket kvar att jobba på. \cite{webEuropeanComission}
Agenda 2030: Globala mål för hållbar utveckling är en ny global utvecklingsagenda framtagen av FN och antagen av världens stats- och regeringschefer som tar vid där millenniemålen slutade år 2015. Denna agenda går in på djupet av problemen mer än vad millenniemålen gjorde. Den fokuserar på hållbar utveckling i alla länder, inte bara u-länder och erkänner arbete mot klimatpåverkan som en grund för att nå en hållbar utveckling. Men kommer denna agenda att lyckas leda världen mot en hållbarare framtid? Det finns både stora förhoppningar på denna utvecklingsplan som bygger på erfarenhet från millenniemålen men även en del kritik. 

\section{Resultat}
Agenda 2030 är en ny global utvecklingsagenda från FN som består av 17 globala mål för hållbar utveckling. \cite{webUNASweden} Agendan\cite{nam2015transforming} är en handlingsplan för människorna, planetesn och vårat välstånd. De globala miljömålen tar vid där millenniemålen slutade och består av 169 delmål för en bättre framtid. I september 2015 i samband med utvärderingen av milleniemålen hos FN i New York skrev samtliga av FNs 193 medlemsstater på Agenda 2030 och de globala målen för hållbar utveckling. \cite{webUNASweden} Målen är universella och gäller både höginkomst- och låginkomstländer \cite{webUNDP}. De Globala målen ska leda världen mot en fredlig och hållbar utveckling till år 2030 och integrerar tre dimensioner av hållbar utveckling: social, ekonomisk och miljömässig.  \cite{UNA Sweden}


 De 17 globala mål som antogs är: 
\begin{enumerate}
\item Utrota fattigdom i alla dess former, överallt.
\item Utrota hunger, uppnå matsäkerhet och förbättrad kost samt främja ett hållbart jordbruk.
\item Säkerställa hälsosamma liv och främja välbefinnande för alla i alla åldrar.
\item Säkerställa inkluderande och rättvis utbildning av god kvalitet och främja livslångt lärande för alla.
\item Uppnå jämställdhet och stärka alla kvinnors och flickors ställning.
\item Säkerställa tillgänglighet och hållbar förvaltning av vatten och sanitet för alla.
\item Säkerställa tillgång till prisvärd, pålitlig, hållbar och modern energi för alla.
\item Främja kontinuerlig, inkluderande och hållbar ekonomisk tillväxt, full och produktiv sysselsättning och anständigt arbete för alla.
\item Bygga motståndskraftig infrastruktur, främja inkluderande och hållbar industrialisering och främja innovation.
\item Minska ojämlikhet inom och mellan länder.
\item Gör städer och boplatser inkluderande, säkra, flexibla och hållbara.
\item Säkerställa hållbara konsumtions- och produktionsmönster.
\item Vidta brådskande åtgärder för att bekämpa klimatförändringarna och dess effekter.
\item Bevara och hållbart nyttja hav, sjöar och marina resurser för hållbar utveckling.
\item Hållbart skogsbruk, stoppa ökenspridning, bromsa och vända markförstöring samt hejda förlusten av biologisk mångfald.
\item Främja fredliga och inkluderande samhällen för hållbar utveckling, ge tillgång till rättssystem för alla och bygga effektiva, ansvarstagande och inkluderande institutioner på alla nivåer.
\item Stärka verktyg för genomförande och vitalisera det globala partnerskapet för hållbar utveckling. \cite{webKTH}
\end{enumerate}



Implementering av de Globala målen och framgång kommer att bero på ländernas egna policys, planer och program. De Globala målen ska fungera som en riktlinje för att styra ländernas planer mot det som överenskommits i Agenda 2030. \cite{web2030agenda}
I Sverige har Sida fått i uppdrag av regeringen att utveckla nya innovativa former för utvecklingsfinansiering. \cite{webSIDA}
Nationellt ledda utvecklingsstrategier kommer att kräva mobilisering av resurser och strategier för finansiering. Regeringen, civilsamhället den privata sektorn och alla andra förväntas samarbeta för att nå de Globala målen. Men även en global gemenskap kommer krävas för att stödja nationella arbeten. 

De Globala målen är däremot inte på något sätt bindande. Länder förväntas själva sätta upp en strategi för att nå den 17 globala målen och implementering kommer att lita på ländernas egna policies och program för hållbar utveckling. Länderna har själva det huvudsakliga ansvaret att följa upp arbetet. Det finns alltså inget som straffar länder som inte gör sin del för att uppnå de Globala målen som undertecknats i Agenda 2030. \cite{web2030agenda}

Att förverkliga dessa mål kommer att kräva stora finansiella insatser och  investering i både U- och I-länder. \cite{web2030agenda}
Hur målen ska implementeras och hur finansiella resurser ska mobiliseras ligger till grund för den nya agendan. http://fn.se/vi-gor/vi-utbildar-och-informerar/fn-info/vad-gor-fn-2/fns-arbete-for-utveckling-och-fattigdomsbekampning/agenda-2030-globala-mal-for-hallbar-utveckling 
Kostnaden beräknas uppgå till trilljoner dollar, \cite{web2030agenda} ca 4500 miljoner USD per år. Detta är trettio gånger mer än det totala årliga biståndet i världen.  https://unicef.se/vad-vi-gor/de-nya-globala-utvecklingsmalen FN:s medlemsländer enades under mötet i New York om ett 100-tal insatser för att finansiera Agenda 2030. Även teknologiöverföring, handel och betydelsen av relevant och bra statistik diskuterades som medel för att uppnå målen. \cite{webUNASweden}
Enligt FN måste resurser mobiliseras från både den kommunala och den privata sektorn, nationellt och internationellt. Det kommer inte att räcka med traditionellt bistånd. Resurserna finns redan i världen men hur investeringar riktas för att stödja hållbar utveckling kommer att spela stor betydelse. Utvecklingshjälp kommer fortfarande att krävas för att hjälpa fattigare länder att nå målen. 
\cite{web2030agenda}
I juli 2015 samlades delegater från hela världen vid FN:s konferens kring utvecklingsfinansiering i Addis Abeba. Resultatet blev Addis Ababa Action Agenda, som fastställer hur de Globala målen ska finansieras.  http://www.sida.se/Svenska/sa-arbetar-vi/Internationellt-samarbete-/globalamalen/
I agendan kom länderna överens en mängd åtgärder för att finansiera de Globala målen såsom förbättrad skatteinsamling och åtgärder mot taxsmitning. Samarbete och officiel utvecklingshjälp mellan länder, speciellt U-länder, var en annan överenskommelse. Agendan understryker även hur viktigt det är med privat investering och incitament från regeringen för att leda dessa investeringar mot hållbar utveckling. Ett nytt sätt att finansiera ny teknologi till U-länder var också med i överenskommelsen. Den innefattade även internationellt samarbete för finansiering av områden där investeringar krävs såsom infrastruktur för energi, transport, vatten och andra områden för att nå de Globala målen.  
\cite{webUNDESA}


Uppföljning och övervakning av målen och arbetet som görs kommer att göras på både global och nationell nivå. Globalt så kommer målen från den nya agendan att mätas mot ett antal globala indikatorer. Dessa indikatorer togs fram av The Inter-Agency and Expert Group on SDG Indicators (IAEG-SDGs) och slogs igenom i mars 2016. \cite{web2030agenda}
De finns idag en llista med 232 indikatorer som följs, se bilagor https://unstats.un.org/sdgs/indicators/indicators-list/.
Regeringar kommer också själva att utveckla nationella indikatorer för att hjälpa till att övervaka framstegen som görs för att nå de 17 målen. Det ska som mål vara 2 indikatorer för varje delmål och ca 300 för alla mål. Uppföljningen av dessa mål kommer att samlas i en rapport för generalsekreteraren i FN. De årliga mötena av  “High-level Political Forum on sustainable development” kommer att spela en central roll i utvärderingen av de Globala målen på global nivå. Implementeringsmetoderna kommer att ses igenom för att försäkra att finansiella resurser är mobiliserade på ett effektivt sätt för att nå de Globala målen.  \cite{web2030agenda}




De Globala målen är en fortsättning på millenniemålen som gick ut år 2015. MIllenniemålen har under sin tid hjälpt till att väcka uppmärksamhet till mängder av problem. Sedan millenniemålen sattes har i världen fattigdom halverats, 90% av barn i U-länder får nu grundläggande utbildning och jämställdhet har ökat. Det finns dock mycket kvar att jobba på. Agenda 2030 bygger på erfarenhet från millenniemålen för att fortsätta jobba mot en hållbar utveckling. \cite{webEuropeanComission},
De Globala målen är mer omfattande än millenniemålen var och adresserar problemen på ett djupare sätt som ska fungera för alla länder. De Globala målen är just globala till skillnad från millenniemålen som var riktade endast mot U-länderna. En viktig del av de Globala målen är att de fokuserar extra mycket på hur de ska implementeras och förverkligas. De nya Globala målen erkänner även att arbete mot klimatförändring är viktigt för hållbar utveckling och bekämpning av fattigdom. \cite{web2030agenda}
Problemen med milleniemålen är att framstegen når ofta inte världens mest utsatta människor. De fattigaste och de som lever i konfliktzoner samt de som diskrimineras på grund av kön, ålder, funktionshinder eller etnicitet. Utvecklingen skiljer sig åt mycket mellan länder och trots framsteg finns mycket kvar att jobba på. 



Men trots de höga förhoppningarna från regeringar som skrivit på agendan så har den fått kritik. Thomas Pogge \cite{critique} på Yales philosophy depatrtmen listar upp flera punkter där de globala målen brister: 

\begin{enumerate}
\item The SDGs promote a false sense of success and make it easy for governments to go
slow on the realization of human rights .
\item The SDGs fail to specify what a human-rights-based duty or genuine goal to eradicate severe poverty requires: a clear division of labor.
\item The full realization of human rights requires a massive roll-back of international and intra-national inequalities, which the SDGs fail to demand.
\end{enumerate}


Eftersom agedan är en överenskommelse mellan länder och inte rättsligt bindande skapar det en möjlighet för länder att fly från sina förpliktelser. Men det skapar även möjlighet att anta en mer ambitiös agenda om så önskas. \cite{critique}


Vad är Agenda 2030: Globala mål för hållbar utveckling? 
Vilka har skrivit på denna agenda?
Vad är kritiken mot agendan? 
Hur sker uppföljningen och övervakningen?
Vad är skillnaden mellan de globala utvecklingsmålen och millenniemålen?
Hur ska arbetet finansieras?
Hur ska Agenda 2030 implementeras? 
Vad händer om ett land inte uppnår målen/ inte följer riktlinjerna? 
Are the Sustainable Development Goals legally binding? 
Vem är ansvarig för att det här sker?
Är det tillräckligt tydliga delmål, hur målen ska uppnås?
Hur gick det för millenniemålen (jämförelse)? 
Har någon utveckling skett sedan Agenda 2030 skapades?



\newpage
\section{Diskussion}

\newpage
\section{Slutsats}
\bibliographystyle{plain}
\bibliography{references} 



\end{document}
