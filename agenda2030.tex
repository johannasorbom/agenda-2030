\documentclass{report}
\usepackage[swedish]{babel}
\usepackage[utf8]{inputenc}
\usepackage{url}

\title{Agenda 2030}
\date{\today}
\author{Johanna Sörbom}

\begin{document}
\maketitle
\newpage
\tableofcontents
\newpage
\section{Inledning}
Hållbar utveckling är i dagens samhälle ett viktigt samtalsämne och något som pushas hårt av bland annat FN. Hållbar utveckling definieras som utveckling som möter dagens behov utan att äventyra framtida generationers möjlighet att uppfylla sina.  \url{http://www.un.org/sustainabledevelopment/development-agenda/}
För att driva utvecklingen framåt på ett hållbart sätt som fungerar för alla världens länder har FNs medlemsländer tidigare kommit fram till millenniemålen som behandlar de viktiga punkterna för hållbarhet och utveckling.  Sedan millenniemålen skapades år 2000 har stora framsteg gjort i fattigdomsbekämpning, jämställdhet och utveckling. Men det finns mycket kvar att jobba på. \url{https://ec.europa.eu/europeaid/policies/european-development-policy/2030-agenda-sustainable-development_en}
Agenda 2030: Globala mål för hållbar utveckling är en ny global utvecklingsagenda framtagen av FN och antagen av världens stats- och regeringschefer som tar vid där millenniemålen slutade år 2015. Denna agenda går in på djupet av problemen mer än vad millenniemålen gjorde. Den fokuserar på hållbar utveckling i alla länder, inte bara u-länder och erkänner arbete mot klimatpåverkan som en grund för att nå en hållbar utveckling. Men kommer denna agenda att lyckas leda världen mot en hållbarare framtid? Det finns både stora förhoppningar på denna utvecklingsplan som bygger på erfarenhet från millenniemålen men även en del kritik. 

\end{document}
