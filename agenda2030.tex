\documentclass{report}
\usepackage[utf8]{inputenc}
\usepackage[swedish]{babel}
\usepackage{url}
\usepackage{hyperref}

\title{Agenda 2030}
\date{\today}
\author{Johanna Sörbom}

\begin{document}
\maketitle
\newpage
\tableofcontents
\newpage
\section{Inledning}
sHållbar utveckling är i dagens samhälle ett viktigt samtalsämne och något som pushas hårt av bland annat FN. Hållbar utveckling definieras som utveckling som möter dagens behov utan att äventyra framtida generationers möjlighet att uppfylla sina.\cite{web2030agenda}
För att driva utvecklingen framåt på ett hållbart sätt som fungerar för alla världens länder har FNs medlemsländer tidigare kommit fram till millenniemålen som behandlar de viktiga punkterna för hållbarhet och utveckling.  Sedan millenniemålen skapades år 2000 har stora framsteg gjort i fattigdomsbekämpning, jämställdhet och utveckling. Men det finns mycket kvar att jobba på. \url{https://ec.europa.eu/europeaid/policies/european-development-policy/2030-agenda-sustainable-development_en}
Agenda 2030: Globala mål för hållbar utveckling är en ny global utvecklingsagenda framtagen av FN och antagen av världens stats- och regeringschefer som tar vid där millenniemålen slutade år 2015. Denna agenda går in på djupet av problemen mer än vad millenniemålen gjorde. Den fokuserar på hållbar utveckling i alla länder, inte bara u-länder och erkänner arbete mot klimatpåverkan som en grund för att nå en hållbar utveckling. Men kommer denna agenda att lyckas leda världen mot en hållbarare framtid? Det finns både stora förhoppningar på denna utvecklingsplan som bygger på erfarenhet från millenniemålen men även en del kritik. 

\section{Resultat}
Vad är Agenda 2030: Globala mål för hållbar utveckling? 
Vilka har skrivit på denna agenda?
Agenda 2030 är en ny global utvecklingsagenda från FN som består av 17 globala mål för hållbar utveckling. 


\url{http://fn.se/vi-gor/vi-utbildar-och-informerar/fn-info/vad-gor-fn-2/fns-arbete-for-utveckling-och-fattigdomsbekampning/agenda-2030-globala-mal-for-hallbar-utveckling} 

Agendan\cite{nam2015transforming} är en handlingsplan för människorna, planetesn och vårat välstånd. De globala miljömålen tar vid där millenniemålen slutade och består av 169 delmål för en bättre framtid. I september 2015 i samband med utvärderingen av milleniemålen hos FN i New York skrev samtliga av FNs 193 medlemsstater på Agenda 2030 och de globala målen för hållbar utveckling. \url{ http://fn.se/vi-gor/vi-utbildar-och-informerar/fn-info/vad-gor-fn-2/fns-arbete-for-utveckling-och-fattigdomsbekampning/agenda-2030-globala-mal-for-hallbar-utveckling} Målen är universella och gäller både höginkomst- och låginkomstländer \url{http://www.globalamalen.se/wp-content/uploads/2016/10/Broschyr-globala-malen.pdf }. De Globala målen ska leda världen mot en fredlig och hållbar utveckling till år 2030 och integrerar tre dimensioner av hållbar utveckling: social, ekonomisk och miljömässig.  \url{http://fn.se/vi-gor/vi-utbildar-och-informerar/fn-info/vad-gor-fn-2/fns-arbete-for-utveckling-och-fattigdomsbekampning/agenda-2030-globala-mal-for-hallbar-utveckling/}
 
 
 De 17 globala mål som antogs är: 
\begin{enumerate}
\item Utrota fattigdom i alla dess former, överallt.
\item Utrota hunger, uppnå matsäkerhet och förbättrad kost samt främja ett hållbart jordbruk.
\item Säkerställa hälsosamma liv och främja välbefinnande för alla i alla åldrar.
\item Säkerställa inkluderande och rättvis utbildning av god kvalitet och främja livslångt lärande för alla.
\item Uppnå jämställdhet och stärka alla kvinnors och flickors ställning.
\item Säkerställa tillgänglighet och hållbar förvaltning av vatten och sanitet för alla.
\item Säkerställa tillgång till prisvärd, pålitlig, hållbar och modern energi för alla.
\item Främja kontinuerlig, inkluderande och hållbar ekonomisk tillväxt, full och produktiv sysselsättning och anständigt arbete för alla.
\item Bygga motståndskraftig infrastruktur, främja inkluderande och hållbar industrialisering och främja innovation.
\item Minska ojämlikhet inom och mellan länder.
\item Gör städer och boplatser inkluderande, säkra, flexibla och hållbara.
\item Säkerställa hållbara konsumtions- och produktionsmönster.
\item Vidta brådskande åtgärder för att bekämpa klimatförändringarna och dess effekter.
\item Bevara och hållbart nyttja hav, sjöar och marina resurser för hållbar utveckling.
\item Hållbart skogsbruk, stoppa ökenspridning, bromsa och vända markförstöring samt hejda förlusten av biologisk mångfald.
\item Främja fredliga och inkluderande samhällen för hållbar utveckling, ge tillgång till rättssystem för alla och bygga effektiva, ansvarstagande och inkluderande institutioner på alla nivåer.
\item Stärka verktyg för genomförande och vitalisera det globala partnerskapet för hållbar utveckling.
\end{enumerate}


\url{https://www.kth.se/om/miljo-hallbar-utveckling/utbildning-miljo-hallbar-utveckling/verktygslada/sustainable-development/globala-mal-for-hallbar-utveckling-1.612059}

\bibliographystyle{plain}
\bibliography{references} 



\end{document}
